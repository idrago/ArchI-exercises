\begin{filecontents*}{tmpcode_1.log}
// CODICE 1:

const int N = ...;
int X[n], Y[N]; // (sizeof(int) = 1 parola)
for (int i=0; i<N; i++)
    Y[i] = X[i] + i;

// CODICE 2:

const int N = ...;
int X[N], Y[N]; // (sizeof(int) = 1 parola)
for (int i=0; i<N; i++)
    Y[i] = X[0] + i;
\end{filecontents*}

\documentclass[11pt]{article}
%\usepackage[handout, nostamp]{moodle}
%\usepackage{catchfilebetweentags}
\usepackage[nostamp]{moodle}
\usepackage{graphicx}
\usepackage{comment}
\usepackage{fancyvrb}
\usepackage{geometry}
\pagestyle{empty}
\geometry{
 a4paper,
 total={175mm,260mm},
 left=15mm,
 top=15mm,
}

\begin{document}

\begin{quiz}{Teoria 1}

\begin{essay}[points=7]{Gerarchia di Memoria e Cache}
Si considerino i seguenti due frammenti di pseudo-codice:

\VerbatimInput[gobble]{tmpcode_1.log}

\begin{itemize}
\item Si illustrino, in generale, i concetti di località spaziale e temporale nell'esecuzione dei programmi.
\item Quale dei due frammenti di codice esibisce maggiore località di ciascun genere?
\item Cosa succede nel caso in cui il processore voglia accedere ad una parola di memoria gi\`{a} presente nella cache? E cosa succede nel caso in cui non sia presente?
\item Quali parametri si devono considerare per valutare le prestazioni delle memorie cache?
\item In un'architettura con memoria cache, quale dei due frammenti di codice risulta in un numero più grande di ``cache hit''?
\end{itemize}

\end{essay}


\begin{essay}[points=7]{I bus}
In riferimento ai bus:
\begin{itemize}
\item Come interagiscono un dispositivo attivo e uno passivo in una comunicazione tramite bus asincrono?
\item Perch\'{e} l'arbitraggio di un bus \`{e} necessario?
\item Si spieghi cos'\`e l'arbitraggio di un bus con meccanismo daisy chain.
\end{itemize}
\end{essay}

\begin{essay}[points=7]{Traduzione programmi}
In riferimento alla traduzione di programmi:
\begin{itemize}
\item Perché non basta una sola lettura del codice sorgente (passata) da parte di un assemblatore?
\item A cosa serve la tabella degli opcode ad un assemblatore?
\item Cosa sono forward e backward references?
\end{itemize}
\end{essay}

\begin{essay}[points=7]{Operazioni di I/O}
In riferimento alle operazioni di Input$/$Output:
\begin{itemize}
\item Cos'è un driver?
\item Descrivere l'I/O basato su DMA.
\item Descrivere l'I/O basato su busy waiting.
\item Descrivere l'I/O basato su interrupt.
\end{itemize}
\end{essay}

\begin{essay}[points=7]{Livello ISA}
In riferimento alle caratteristiche del livello ISA. Descrivere:
\begin{itemize}
\item Cosa si intende per ``modalità di indirizzamento''?
\item La differenza tra ordinamento little endian e big endian dei byte all'interno di una parola;
\item Nell'architettura RISC-V, quali sono i tipi di registri? Perché ci sono diversi tipi?
\end{itemize}
\end{essay}

\begin{essay}[points=7]{RISC-V}
Si consideri l'architettura RISC-V.

\begin{itemize}
\item Si descriva il ciclo di aggiornamento del Program Counter.
\item Si spieghi come deve essere configurato il datapath (percorso dei dati) per l'esecuzione dell'operazione ``beq x11, x12, LABEL''.
\end{itemize}
\end{essay}

\end{quiz}
\end{document}



