\documentclass[11pt]{article}
%\usepackage[handout, nostamp]{moodle}
%\usepackage{catchfilebetweentags}
\usepackage[nostamp]{moodle}
\usepackage{graphicx}
\usepackage{comment}
\usepackage{fancyvrb}
\usepackage{geometry}
\pagestyle{empty}
 \geometry{
 a4paper,
 total={175mm,260mm},
 left=15mm,
 top=15mm,
 }

\begin{document}

\begin{quiz}{Rappresentazione numerica}

\begin{cloze}[points=1,shuffle=false]{Passaggio tra basi}
Come si scriverebbe la sequenza di 6 cifre binarie $A = 111000$ in base:

\begin{itemize}
\item 4: \begin{shortanswer}\item 320\end{shortanswer}
\item 8: \begin{shortanswer}\item 70\end{shortanswer}
\item 16: \begin{shortanswer}\item 38\end{shortanswer}
\item 5: \begin{shortanswer}\item 211\end{shortanswer}
\end{itemize}
\end{cloze}

\begin{cloze}[points=1,shuffle=false]{Passaggio tra basi}
Come si scriverebbe la sequenza $400$ in base 5 in base:

\begin{itemize}
\item 3: \begin{shortanswer}\item 10201\end{shortanswer}
\item 7: \begin{shortanswer}\item 202\end{shortanswer}
\item 9: \begin{shortanswer}\item 121\end{shortanswer}
\end{itemize}

Come si scriverebbe la sequenza $400$ in base 6 in base:

\begin{itemize}
\item 7: \begin{shortanswer}\item 264\end{shortanswer}
\end{itemize}
\end{cloze}

\begin{cloze}[points=1,shuffle=false]{Codifica numeri relativi}
Data la sequenza di 6 bit $A = 110110$ quale numero relativo sarebbe rappresentato se la codifica fosse:

\begin{itemize}
\item In modulo e segno: \begin{shortanswer}\item -22\end{shortanswer}
\item In complemento a uno: \begin{shortanswer}\item -9\end{shortanswer}
\item In complemento a due: \begin{shortanswer}\item -10\end{shortanswer}
\item In eccesso $2^5$: \begin{shortanswer}\item 22\end{shortanswer}
\item In eccesso $2^3$: \begin{shortanswer}\item 46\end{shortanswer}
\end{itemize}
\end{cloze}

\begin{cloze}[points=1,shuffle=false]{Codifica numeri relativi}
Data la sequenza di 6 bit $A = 111111$ quale numero relativo sarebbe rappresentato se la codifica fosse:

\begin{itemize}
\item In modulo e segno: \begin{shortanswer}\item -31\end{shortanswer}
\item In complemento a uno: \begin{shortanswer}\item -0\end{shortanswer}
\item In complemento a due: \begin{shortanswer}\item -1\end{shortanswer}
\item In eccesso $2^5$: \begin{shortanswer}\item 31\end{shortanswer}
\item In eccesso $2^3$: \begin{shortanswer}\item 55\end{shortanswer}
\end{itemize}

\end{cloze}

\begin{cloze}[points=1,shuffle=false]{Codifica numeri relativi}
Dato il numero -20 quale sarebbe la sua codifica su 6 bit nel caso si usasse:

\begin{itemize}
\item In modulo e segno: \begin{shortanswer}\item 110100\end{shortanswer}
\item In complemento a uno: \begin{shortanswer}\item 101011\end{shortanswer}
\item In complemento a due: \begin{shortanswer}\item 101100\end{shortanswer}
\item In eccesso $21$: \begin{shortanswer}\item 000001\end{shortanswer}
\end{itemize}
\end{cloze}

\begin{cloze}[points=1,shuffle=false]{Codifica numeri relativi}
Dato il numero -21 quale sarebbe la sua codifica su 6 bit nel caso si usasse:

\begin{itemize}
\item In modulo e segno: \begin{shortanswer}\item 110101\end{shortanswer}
\item In complemento a uno: \begin{shortanswer}\item 101010\end{shortanswer}
\item In complemento a due: \begin{shortanswer}\item 101011\end{shortanswer}
\item In eccesso $23$: \begin{shortanswer}\item 000010\end{shortanswer}
\end{itemize}
\end{cloze}

\begin{cloze}[points=1,shuffle=false]{Complemento a 2}
Quale numero decimale è rappresentato dalla combinazione di bit 0x0C000000 nel caso in cui si tratti di un...
\begin{itemize}
\item numero intero in complemento a 2? \begin{shortanswer}\item 201326592\end{shortanswer}
\item numero intero senza segno? \begin{shortanswer}\item 201326592\end{shortanswer}
\end{itemize}
\end{cloze}

\begin{cloze}[points=1,shuffle=false]{Complemento a 2}
Indicare quale \`{e} il numero minimo di bit necessari per rappresentare in complemento a due i numeri A = (+47) e B = (-3). Riportare la codifica in binario dei due numeri utilizzando lo stesso numero minimo di bit.

\begin{itemize}
\item Servono \begin{shortanswer}\item 7\end{shortanswer} bit per rappresentare $A$
\item Servono \begin{shortanswer}\item 3\end{shortanswer} bit per rappresentare $B$
\item Codifica di $A$ \begin{shortanswer}\item 0101111\end{shortanswer}
\item Codifica di $B$ \begin{shortanswer}\item 101\end{shortanswer}
\end{itemize}
\end{cloze}

\begin{cloze}[points=1,shuffle=false]{Complemento a 2}
Indicare quale \`{e} il numero minimo di bit necessari per rappresentare in complemento a due i numeri A = (+3) e B = (-31). Riportare la codifica in binario dei due numeri utilizzando lo stesso numero minimo di bit.

\begin{itemize}
\item Servono \begin{shortanswer}\item 3\end{shortanswer} bit per rappresentare $A$
\item Servono \begin{shortanswer}\item 6\end{shortanswer} bit per rappresentare $B$
\item Codifica di $A$ \begin{shortanswer}\item 011\end{shortanswer}
\item Codifica di $B$ \begin{shortanswer}\item 100001\end{shortanswer}
\end{itemize}
\end{cloze}

\begin{cloze}[points=1,shuffle=false]{Complemento a 2}
Indicare quale \`{e} il numero minimo di bit necessari per rappresentare in complemento a due i numeri A = (+47) e B = (-11). Riportare la codifica in binario dei due numeri utilizzando lo stesso numero minimo di bit.

\begin{itemize}
\item Servono \begin{shortanswer}\item 7\end{shortanswer} bit per rappresentare $A$
\item Servono \begin{shortanswer}\item 5\end{shortanswer} bit per rappresentare $B$
\item Codifica di $A$ \begin{shortanswer}\item 0101111\end{shortanswer}
\item Codifica di $B$ \begin{shortanswer}\item 10101\end{shortanswer}
\end{itemize}
\end{cloze}

\begin{multi}[points=1]{Complemento a 2}
Si consideri i seguenti numeri binari in complemento a 2, \`{e} vero che:

$000010110001 > 000011101000$

\item Vero
\item* Falso
\end{multi}

\begin{multi}[points=1]{Complemento a 2}
Si consideri i seguenti numeri binari in complemento a 2, \`{e} vero che:

$000011101000 > 000010110001$

\item* Vero
\item Falso
\end{multi}

\begin{multi}[points=1]{Complemento a 2}
Si consideri i seguenti numeri binari in complemento a 2, \`{e} vero che:

$111101001111 > 111100011000$

\item* Vero
\item Falso
\end{multi}

\begin{multi}[points=1]{Complemento a 2}
Si consideri i seguenti numeri binari in complemento a 2, \`{e} vero che:

$111100011000 > 111101001111$

\item Vero
\item* Falso
\end{multi}



\begin{cloze}[points=1,shuffle=false]{Complemento a uno}
Si supponga che $A = 000001$ e $B = 110110$ siano le rappresentazioni in \textbf{complemento a uno} di due numeri in un elaboratore che esprime gli interi su 6 bit.

Scrivere le rappresentazioni di $A+B$, usando sempre 6 bit:

\begin{itemize}
\item In modulo e segno: \begin{shortanswer}\item 101000\end{shortanswer}
\item In complemento a due: \begin{shortanswer}\item 111000\end{shortanswer}
\item In eccesso $2^5$: \begin{shortanswer}\item 011000\end{shortanswer}
\end{itemize}
\end{cloze}

\begin{cloze}[points=1,shuffle=false]{Complemento a uno}
Si supponga che $A = 000110$ e $B = 110011$ siano le rappresentazioni in \textbf{complemento a uno} di due numeri in un elaboratore che esprime gli interi su 6 bit.

Scrivere le rappresentazioni di $A+B$, usando sempre 6 bit:

\begin{itemize}
\item In modulo e segno: \begin{shortanswer}\item 100110\end{shortanswer}
\item In complemento a due: \begin{shortanswer}\item 111010\end{shortanswer}
\item In eccesso $2^5$: \begin{shortanswer}\item 011010\end{shortanswer}
\end{itemize}
\end{cloze}

\begin{cloze}[points=1,shuffle=false]{Numeri binari}
Un elaboratore esprime gli interi su 16 bit. Scrivere le rappresentazioni in binario puro dei numeri:

\begin{itemize}
\item $256_{10}$ \begin{shortanswer}\item 0000000100000000\end{shortanswer}
\item $10_{10}$  \begin{shortanswer}\item 0000000000001010\end{shortanswer}
\item $27_{10}$  \begin{shortanswer}\item 0000000000011011 \end{shortanswer}
\item $32768_{10}$ \begin{shortanswer}\item 1000000000000000\end{shortanswer}
\end{itemize}
\end{cloze}

\begin{cloze}[points=1,shuffle=false]{Da esadecimali a binario}
Convertire in binario:
\begin{itemize}
\item 5ED4 \begin{shortanswer}\item 0101111011010100\end{shortanswer}
\item FFFF \begin{shortanswer}\item 1111111111111111\end{shortanswer}
\item CAFE \begin{shortanswer}\item 1100101011111110\end{shortanswer}
\item 0BAD \begin{shortanswer}\item 0000101110101101\end{shortanswer}
\item FACE \begin{shortanswer}\item 1111101011001110\end{shortanswer}
\item B00C \begin{shortanswer}\item 1011000000001100\end{shortanswer}
\end{itemize}
\end{cloze}

\begin{essay}[points=1, feedback={L'attrattiva è che ogni cifra esadecimale rappresenta uno dei 16 caratteri (0-9, A-E). Poiché con 4 bit è possibile rappresentare 16 elementi, in esadecimale ogni cifra richiede esattamente 4 bit. E 1 byte è lungo 8 bit, quindi sono sufficienti due cifre esadecimali per rappresentare il contenuto di 1 byte.}]{Esadecimale}
Che cosa rende la base 16 (esadecimale) un sistema di numerazione così attraente per rappresentare i numeri in un calcolatore?
\end{essay}

\begin{cloze}[points=1,shuffle=false]{Esadecimale e ottale}
Convertire i seguenti numeri binari...

A) in esadecimale:
\begin{itemize}
\item $101101100010$ \begin{shortanswer}\item B62\end{shortanswer}
\item $101110101010110111$ \begin{shortanswer}\item  2EAB7\end{shortanswer}
\end{itemize}

B) in ottale:
\begin{itemize}
\item $101101100010$ \begin{shortanswer}\item 5542\end{shortanswer}
\item $101110101010110111$ \begin{shortanswer}\item 565267\end{shortanswer}
\end{itemize}
\end{cloze}

\begin{cloze}[points=1,shuffle=false]{Esadecimale e ottale}
Sia dato il numero binario frazionario 101110000,101. Convertirlo in:

\begin{itemize}
\item base 8 \begin{shortanswer}\item 560,5\end{shortanswer}
\item base 16 \begin{shortanswer}\item 170,A\end{shortanswer}
\item base 10 \begin{shortanswer}\item 368,625\end{shortanswer}
\end{itemize}
\end{cloze}

\begin{cloze}[points=1,shuffle=false]{Vero o Falso}
Indicare se le seguenti affermazioni sono vere o false.

Con 8 bit è possibile rappresentare:
\begin{itemize}
\item tutti gli interi non negativi minori o uguali a 255 in binario puro.\begin{multi}[inline]\item* Vero \item Falso\end{multi}
\item tutti gli interi non negativi minori o uguali a 255 in modulo e segno.\begin{multi}[inline]\item Vero \item* Falso\end{multi}
\item tutti gli interi compresi nell'intervallo $[-256,+255]$ in complemento a due.\begin{multi}[inline]\item Vero \item* Falso\end{multi}
\item tutti gli interi compresi nell'intervallo $[-127,+127]$ in complemento a uno.\begin{multi}[inline]\item* Vero \item Falso\end{multi}
\end{itemize}
\end{cloze}

\begin{cloze}[points=1,shuffle=false]{Esadecimali}
\begin{itemize}
\item Qual è il risultato della sottrazione 5ED4 – 07A4 se questi numeri sono rappresentati come numeri esadecimali senza segno su 16 bit? Scrivere il risultato in esadecimale.\begin{shortanswer}\item 5730\end{shortanswer}
\item Qual è il risultato della sottrazione 5ED4 – 07A4 se questi numeri sono rappresentati come numeri esadecimali dotati di segno su 16 bit, codificati in modulo e segno? Scrivere il risultato in esadecimale.\begin{shortanswer}\item 5730\end{shortanswer}
\end{itemize}
\end{cloze}

\begin{cloze}[points=1,shuffle=false]{Ottali}
Scrivere il risultato in ottale.
\begin{itemize}
\item Qual è il risultato di 4365 – 3412 se essi sono rappresentati come numeri ottali senza segno su 12 bit? \begin{shortanswer}\item 0753\end{shortanswer}
    \item Qual è il risultato di 4365 – 3412 se essi sono rappresentati come numeri dotati di segno su 12 bit, codificati in modulo e segno? \begin{shortanswer}\item -3777 \item 7777\end{shortanswer}
\end{itemize}
\end{cloze}

\begin{cloze}[points=1,shuffle=false]{Overflow}
\begin{itemize}
\item Si supponga che 127 e 122 siano numeri decimali interi senza segno su 8 bit. Calcolare 127 + 122 e dichiarare se si verifica overflow. \begin{multi}[inline]\item Overflow \item* No overflow\end{multi}
\item Si supponga che 127 e 122 siano numeri decimali interi dotati di segno su 8 bit, codificati in modulo e segno. Calcolare 127 + 122 e dichiarare se si verifica overflow. \begin{multi}[inline]\item* Overflow \item No overflow\end{multi}
\item Si supponga che -127 e 122 siano numeri decimali interi dotati di segno su 8 bit, codificati in modulo e segno. Calcolare -127 – 122 e dichiarare se si verifica overflow.\begin{multi}[inline]\item* Overflow \item No overflow\end{multi}
\end{itemize}
\end{cloze}

\begin{cloze}[points=1,shuffle=false]{Overflow}
Siano dati a=11110101, b=11101010, due numeri binari su 8 bit.
Calcolare il risultato della somma (a + b) scrivendo il valore ottenuto in decimale e indicare la presenza di un eventuale overflow considerando gli operandi codificati rispettivamente come:
\begin{itemize}
\item numeri interi positivi
\begin{shortanswer}\item 479\end{shortanswer}
\begin{multi}[inline]\item* Overflow \item No overflow\end{multi}
%
\item numeri interi con segno rappresentati in modulo e segno
\begin{shortanswer}\item -223\end{shortanswer}
\begin{multi}[inline]\item* Overflow \item No overflow\end{multi}
%
\item numeri interi con segno rappresentati in complemento a 1
\begin{shortanswer}\item -31\end{shortanswer}
\begin{multi}[inline]\item Overflow \item* No overflow\end{multi}
%
\item numeri interi con segno rappresentati in complemento a 2
\begin{shortanswer}\item -33\end{shortanswer}
\begin{multi}[inline]\item Overflow \item* No overflow\end{multi}
\end{itemize}
\end{cloze}

% \begin{itemize}
% \item
% \item a=11001010, b=01001100
% \end{itemize}



\begin{cloze}[points=1,shuffle=false]{Complemento a 2}
Una caratteristica che in genere non si trova nei microprocessori a utilizzo generale è la saturazione. Saturazione significa che quando si
verifica un overflow, il risultato assume il valore del numero positivo o negativo più grande rappresentabile, invece del valore del risultato calcolato nell'aritmetica in complemento a 2. La saturazione è ampiamente utilizzata nelle operazioni multimediali. Per esempio, sarebbe fastidioso se girando la manopola del volume della radio il volume aumentasse man mano per diventare poi improvvisamente bassissimo. Con un meccanismo di saturazione, il dispositivo, una volta raggiunto il volume massimo, continuerebbe a riprodurre il suono a quel volume anche se si
continua a girare la manopola. Le estensioni multimediali degli insiemi standard di istruzioni prevedono spesso l'aritmetica con saturazione.

\begin{itemize}
\item Si supponga che A=0b11010110 e B=0b10010111 siano numeri interi dotati di segno su 8 bit, codificati in complemento a 2. Calcolare A + B utilizzando l'aritmetica con saturazione e scrivere il risultato in notazione decimale.\begin{shortanswer}\item -128\end{shortanswer}
\item Si supponga che A=0b11010110 e B=0b10010111 siano numeri decimali interi dotati di segno su 8 bit, codificati in complemento a 2. Calcolare A - B utilizzando l'aritmetica con saturazione e scrivere il risultato in notazione decimale.\begin{shortanswer}\item 63\end{shortanswer}
\item Si supponga che 151 e 214 siano numeri decimali interi senza segno su 8 bit. Calcolare 151 + 214 utilizzando l'aritmetica con saturazione e scrivere il risultato in notazione decimale.\begin{shortanswer}\item 255\end{shortanswer}
\end{itemize}
\end{cloze}

\end{quiz}
\end{document}



