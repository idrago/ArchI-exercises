\documentclass[11pt]{article}
%\usepackage[handout, nostamp]{moodle}
%\usepackage{catchfilebetweentags}
\usepackage[nostamp]{moodle}
\usepackage{graphicx}
\usepackage{comment}
\usepackage{fancyvrb}
\usepackage{geometry}
\pagestyle{empty}
 \geometry{
 a4paper,
 total={175mm,260mm},
 left=15mm,
 top=15mm,
 }

\begin{document}

\begin{quiz}{Rappresentazione numerica}

%#######################
\begin{cloze}[points=1,shuffle=false]{Passaggio tra basi}
Come si scriverebbe la sequenza di 6 cifre binarie $A = 111000$ in base:
\begin{itemize}
\item 4: \begin{numerical}\item 320\end{numerical}
\item 8: \begin{numerical}\item 70\end{numerical}
\item 16: \begin{numerical}\item 38\end{numerical}
\item 5: \begin{numerical}\item 211\end{numerical}
\end{itemize}
\end{cloze}

%#######################
\begin{cloze}[points=1,shuffle=false]{Passaggio tra basi}
Come si scriverebbe la sequenza $400$ in base 5 in base:
\begin{itemize}
\item 3: \begin{numerical}\item 10201\end{numerical}
\item 7: \begin{numerical}\item 202\end{numerical}
\item 9: \begin{numerical}\item 121\end{numerical}
\end{itemize}

Come si scriverebbe la sequenza $400$ in base 6 in base:
\begin{itemize}
\item 7: \begin{numerical}\item 264\end{numerical}
\end{itemize}
\end{cloze}

%#######################
\begin{cloze}[points=1,shuffle=false]{Codifica numeri relativi}
Data la sequenza di 6 bit $A = 110110$ quale numero relativo sarebbe rappresentato se la codifica fosse:
\begin{itemize}
\item In modulo e segno: \begin{numerical}\item -22\end{numerical}
\item In complemento a uno: \begin{numerical}\item -9\end{numerical}
\item In complemento a due: \begin{numerical}\item -10\end{numerical}
\item In eccesso $2^5$: \begin{numerical}\item 22\end{numerical}
\item In eccesso $2^3$: \begin{numerical}\item 46\end{numerical}
\end{itemize}
\end{cloze}

%#######################
\begin{cloze}[points=1,shuffle=false]{Codifica numeri relativi}
Data la sequenza di 6 bit $A = 111111$ quale numero relativo sarebbe rappresentato se la codifica fosse:
\begin{itemize}
\item In modulo e segno: \begin{numerical}\item -31\end{numerical}
\item In complemento a uno: \begin{numerical}\item -0\end{numerical}
\item In complemento a due: \begin{numerical}\item -1\end{numerical}
\item In eccesso $2^5$: \begin{numerical}\item 31\end{numerical}
\item In eccesso $2^3$: \begin{numerical}\item 55\end{numerical}
\end{itemize}
\end{cloze}

%#######################
\begin{cloze}[points=1,shuffle=false]{Codifica numeri relativi}
Dato il numero -20 quale sarebbe la sua codifica su 6 bit nel caso si usasse:
\begin{itemize}
\item In modulo e segno: \begin{shortanswer}\item 110100\end{shortanswer}
\item In complemento a uno: \begin{shortanswer}\item 101011\end{shortanswer}
\item In complemento a due: \begin{shortanswer}\item 101100\end{shortanswer}
\item In eccesso $21$: \begin{shortanswer}\item 000001\end{shortanswer}
\end{itemize}
\end{cloze}

%#######################
\begin{cloze}[points=1,shuffle=false]{Codifica numeri relativi}
Dato il numero -21 quale sarebbe la sua codifica su 6 bit nel caso si usasse:
\begin{itemize}
\item In modulo e segno: \begin{shortanswer}\item 110101\end{shortanswer}
\item In complemento a uno: \begin{shortanswer}\item 101010\end{shortanswer}
\item In complemento a due: \begin{shortanswer}\item 101011\end{shortanswer}
\item In eccesso $23$: \begin{shortanswer}\item 000010\end{shortanswer}
\end{itemize}
\end{cloze}

%#######################
\begin{cloze}[points=1,shuffle=false]{Complemento a 2}
Quale numero decimale è rappresentato dalla combinazione di bit 0x0C000000 nel caso in cui si tratti di un...
\begin{itemize}
\item numero intero in complemento a 2? \begin{numerical}\item 201326592\end{numerical}
\item numero intero senza segno? \begin{numerical}\item 201326592\end{numerical}
\end{itemize}
\end{cloze}


%#######################
\begin{cloze}[points=1,shuffle=false]{Complemento a 2}
Indicare quale \`{e} il numero minimo di bit necessari per rappresentare in complemento a due i numeri A = (+47) e B = (-3). Riportare la codifica in binario dei due numeri utilizzando lo stesso numero minimo di bit.

\begin{itemize}
\item Servono \begin{numerical}\item 7\end{numerical} bit per rappresentare $A$
\item Servono \begin{numerical}\item 3\end{numerical} bit per rappresentare $B$
\item Codifica di $A$ \begin{shortanswer}\item 0101111\end{shortanswer}
\item Codifica di $B$ \begin{shortanswer}\item 101\end{shortanswer}
\end{itemize}
\end{cloze}

%#######################
\begin{cloze}[points=1,shuffle=false]{Complemento a 2}
Indicare quale \`{e} il numero minimo di bit necessari per rappresentare in complemento a due i numeri A = (+3) e B = (-31). Riportare la codifica in binario dei due numeri utilizzando lo stesso numero minimo di bit.

\begin{itemize}
\item Servono \begin{numerical}\item 3\end{numerical} bit per rappresentare $A$
\item Servono \begin{numerical}\item 6\end{numerical} bit per rappresentare $B$
\item Codifica di $A$ \begin{shortanswer}\item 011\end{shortanswer}
\item Codifica di $B$ \begin{shortanswer}\item 100001\end{shortanswer}
\end{itemize}
\end{cloze}

%#######################
\begin{cloze}[points=1,shuffle=false]{Complemento a 2}
Indicare quale \`{e} il numero minimo di bit necessari per rappresentare in complemento a due i numeri A = (+47) e B = (-11). Riportare la codifica in binario dei due numeri utilizzando lo stesso numero minimo di bit.

\begin{itemize}
\item Servono \begin{numerical}\item 7\end{numerical} bit per rappresentare $A$
\item Servono \begin{numerical}\item 5\end{numerical} bit per rappresentare $B$
\item Codifica di $A$ \begin{shortanswer}\item 0101111\end{shortanswer}
\item Codifica di $B$ \begin{shortanswer}\item 10101\end{shortanswer}
\end{itemize}
\end{cloze}

%#######################
\begin{multi}[points=1]{Complemento a 2}
Si consideri i seguenti numeri binari in complemento a 2, \`{e} vero che:

$000010110001 > 000011101000$

\item Vero
\item* Falso
\end{multi}


%#######################
\begin{multi}[points=1]{Complemento a 2}
Si consideri i seguenti numeri binari in complemento a 2, \`{e} vero che:

$000011101000 > 000010110001$

\item* Vero
\item Falso
\end{multi}


%#######################
\begin{multi}[points=1]{Complemento a 2}
Si consideri i seguenti numeri binari in complemento a 2, \`{e} vero che:

$111101001111 > 111100011000$

\item* Vero
\item Falso
\end{multi}


%#######################
\begin{multi}[points=1]{Complemento a 2}
Si consideri i seguenti numeri binari in complemento a 2, \`{e} vero che:

$111100011000 > 111101001111$

\item Vero
\item* Falso
\end{multi}


%#######################
\begin{cloze}[points=1,shuffle=false]{Complemento a uno}
Si supponga che $A = 000001$ e $B = 110110$ siano le rappresentazioni in \textbf{complemento a uno} di due numeri in un elaboratore che esprime gli interi su 6 bit.

Scrivere le rappresentazioni di $A+B$, usando sempre 6 bit:

\begin{itemize}
\item In modulo e segno: \begin{shortanswer}\item 101000\end{shortanswer}
\item In complemento a due: \begin{shortanswer}\item 111000\end{shortanswer}
\item In eccesso $2^5$: \begin{shortanswer}\item 011000\end{shortanswer}
\end{itemize}
\end{cloze}

%#######################
\begin{cloze}[points=1,shuffle=false]{Complemento a uno}
Si supponga che $A = 000110$ e $B = 110011$ siano le rappresentazioni in \textbf{complemento a uno} di due numeri in un elaboratore che esprime gli interi su 6 bit.

Scrivere le rappresentazioni di $A+B$, usando sempre 6 bit:

\begin{itemize}
\item In modulo e segno: \begin{shortanswer}\item 100110\end{shortanswer}
\item In complemento a due: \begin{shortanswer}\item 111010\end{shortanswer}
\item In eccesso $2^5$: \begin{shortanswer}\item 011010\end{shortanswer}
\end{itemize}
\end{cloze}

%#######################
\begin{cloze}[points=1,shuffle=false]{Numeri binari}
Un elaboratore esprime gli interi su 16 bit. Scrivere le rappresentazioni in binario puro dei numeri:

\begin{itemize}
\item $256_{10}$ \begin{shortanswer}\item 0000000100000000\end{shortanswer}
\item $10_{10}$  \begin{shortanswer}\item 0000000000001010\end{shortanswer}
\item $27_{10}$  \begin{shortanswer}\item 0000000000011011 \end{shortanswer}
\item $32768_{10}$ \begin{shortanswer}\item 1000000000000000\end{shortanswer}
\end{itemize}
\end{cloze}

%#######################
\begin{cloze}[points=1,shuffle=false]{Da esadecimali a binario}
Convertire in binario:
\begin{itemize}
\item 5ED4 \begin{shortanswer}\item 0101111011010100\end{shortanswer}
\item FFFF \begin{shortanswer}\item 1111111111111111\end{shortanswer}
\item CAFE \begin{shortanswer}\item 1100101011111110\end{shortanswer}
\item 0BAD \begin{shortanswer}\item 0000101110101101\end{shortanswer}
\item FACE \begin{shortanswer}\item 1111101011001110\end{shortanswer}
\item B00C \begin{shortanswer}\item 1011000000001100\end{shortanswer}
\end{itemize}
\end{cloze}

%#######################
\begin{essay}[points=1, feedback={L'attrattiva è che ogni cifra esadecimale rappresenta uno dei 16 caratteri (0-9, A-E). Poiché con 4 bit è possibile rappresentare 16 elementi, in esadecimale ogni cifra richiede esattamente 4 bit. E 1 byte è lungo 8 bit, quindi sono sufficienti due cifre esadecimali per rappresentare il contenuto di 1 byte.}]{Esadecimale}
Che cosa rende la base 16 (esadecimale) un sistema di numerazione così attraente per rappresentare i numeri in un calcolatore?
\end{essay}

%#######################
\begin{cloze}[points=1,shuffle=false]{Esadecimale e ottale}
Convertire i seguenti numeri binari...

A-- in esadecimale:
\begin{itemize}
\item $101101100010$ \begin{shortanswer}\item B62\end{shortanswer}
\item $101110101010110111$ \begin{shortanswer}\item  2EAB7\end{shortanswer}
\end{itemize}

B-- in ottale:
\begin{itemize}
\item $101101100010$ \begin{shortanswer}\item 5542\end{shortanswer}
\item $101110101010110111$ \begin{shortanswer}\item 565267\end{shortanswer}
\end{itemize}
\end{cloze}

%#######################
\begin{cloze}[points=1,shuffle=false]{Esadecimale e ottale}
Sia dato il numero binario frazionario 101110000,101. Convertirlo in:

\begin{itemize}
\item base 8 \begin{shortanswer}\item 560,5\end{shortanswer}
\item base 16 \begin{shortanswer}\item 170,A\end{shortanswer}
\item base 10 \begin{shortanswer}\item 368,625\end{shortanswer}
\end{itemize}
\end{cloze}

%#######################
\begin{cloze}[points=1,shuffle=false]{Vero o Falso}
Indicare se le seguenti affermazioni sono vere o false.

Con 8 bit è possibile rappresentare:
\begin{itemize}
\item tutti gli interi non negativi minori o uguali a 255 in binario puro.\begin{multi}[inline]\item* Vero \item Falso\end{multi}
\item tutti gli interi non negativi minori o uguali a 255 in modulo e segno.\begin{multi}[inline]\item Vero \item* Falso\end{multi}
\item tutti gli interi compresi nell'intervallo $[-256,+255]$ in complemento a due.\begin{multi}[inline]\item Vero \item* Falso\end{multi}
\item tutti gli interi compresi nell'intervallo $[-127,+127]$ in complemento a uno.\begin{multi}[inline]\item* Vero \item Falso\end{multi}
\end{itemize}
\end{cloze}

%#######################
\begin{cloze}[points=1,shuffle=false]{Esadecimali}
\begin{itemize}
\item Qual è il risultato della sottrazione 5ED4 - 07A4 se questi numeri sono rappresentati come numeri esadecimali senza segno su 16 bit? Scrivere il risultato in esadecimale.
\begin{shortanswer}
    \item 5730
\end{shortanswer}
\item Qual è il risultato della sottrazione 5ED4 - 07A4 se questi numeri sono rappresentati come numeri esadecimali dotati di segno su 16 bit, codificati in modulo e segno? Scrivere il risultato in esadecimale.
\begin{shortanswer}
    \item 5730
\end{shortanswer}
\end{itemize}
\end{cloze}

%#######################
\begin{cloze}[points=1,shuffle=false]{Ottali}
Scrivere il risultato in ottale.
\begin{itemize}
\item Qual è il risultato di 4365 - 3412 se essi sono rappresentati come numeri ottali senza segno su 12 bit? 
\begin{shortanswer}
    \item 0753
\end{shortanswer}
\item Qual è il risultato di 4365 - 3412 se essi sono rappresentati come numeri dotati di segno su 12 bit, codificati in modulo e segno? 
\begin{shortanswer}
    \item -3777 
    \item 7777
\end{shortanswer}
\end{itemize}
\end{cloze}

\end{quiz}
\end{document}



