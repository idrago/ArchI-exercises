\documentclass[11pt]{article}
%\usepackage[handout, nostamp]{moodle}
%\usepackage{catchfilebetweentags}
\usepackage[nostamp]{moodle}
\usepackage{graphicx}
\usepackage{comment}
\usepackage{fancyvrb}
\usepackage{geometry}
\pagestyle{empty}
 \geometry{
 a4paper,
 total={175mm,260mm},
 left=15mm,
 top=15mm,
 }

\begin{document}
\begin{quiz}{Numeri in virgola mobile}

\begin{cloze}[points=1,shuffle=false]{Codifica dei numeri in virgola mobile}
Come verrebbe rappresentato il numero decimale -11.75 nell'inesistente codifica IEEE 754 in precisione
insoddisfacente che utilizza 1 bit per il segno, 4 bit per l'esponente e 5 bit per la mantissa?
\begin{itemize}
\item bit di segno: \begin{shortanswer}\item 1 \end{shortanswer}
\item esponente: \begin{shortanswer}\item 1010 \end{shortanswer}
\item mantissa: \begin{shortanswer}\item 01111 \end{shortanswer}
\end{itemize}
\end{cloze}

\begin{cloze}[points=1,shuffle=false]{Codifica dei numeri in virgola mobile}
Come verrebbe rappresentato il numero decimale -5.0 nell'inesistente codifica IEEE 754 in precisione
insoddisfacente che utilizza 1 bit per il segno, 4 bit per l'esponente e 5 bit per la mantissa?
\begin{itemize}
\item bit di segno: \begin{shortanswer}\item 1 \end{shortanswer}
\item esponente: \begin{shortanswer}\item 1001 \end{shortanswer}
\item mantissa: \begin{shortanswer}\item 01000 \end{shortanswer}
\end{itemize}
\end{cloze}

\begin{cloze}[points=1,shuffle=false]{Codifica dei numeri in virgola mobile}
Come verrebbe rappresentato il numero decimale 7.5 nell'inesistente codifica IEEE 754 in precisione
insoddisfacente che utilizza 1 bit per il segno, 3 bit per l'esponente e 3 bit per la mantissa?
\begin{itemize}
\item bit di segno: \begin{shortanswer}\item 0\end{shortanswer}
\item esponente: \begin{shortanswer}\item 101\end{shortanswer} 
\item mantissa: \begin{shortanswer}\item 111\end{shortanswer}
\end{itemize}
\end{cloze}

\begin{cloze}[points=1,shuffle=false]{Codifica dei numeri in virgola mobile}
Come verrebbe rappresentato il numero decimale -11 nell'inesistente codifica IEEE 754 in precisione
insoddisfacente che utilizza 1 bit per il segno, 3 bit per l'esponente e 3 bit per la mantissa?
\begin{itemize}
\item bit di segno: \begin{shortanswer}\item 1\end{shortanswer}
\item esponente: \begin{shortanswer}\item 1101\end{shortanswer} 
\item mantissa: \begin{shortanswer}\item 011\end{shortanswer}
\end{itemize}
\end{cloze}

\begin{cloze}[points=1,shuffle=false]{Codifica dei numeri in virgola mobile}
Come verrebbe rappresentato il numero decimale 34 nell'inesistente codifica IEEE 754 in precisione
insoddisfacente che utilizza 1 bit per il segno, 4 bit per l'esponente e 4 bit per la mantissa?
\begin{itemize}
\item bit di segno: \begin{shortanswer}\item 0\end{shortanswer}
\item esponente: \begin{shortanswer}\item 1100\end{shortanswer} 
\item mantissa: \begin{shortanswer}\item 0001\end{shortanswer}
\end{itemize}
\end{cloze}

\begin{cloze}[points=1,shuffle=false]{Codifica dei numeri in virgola mobile}
Come verrebbe rappresentato il numero decimale -3.25 nell'inesistente codifica IEEE 754 in precisione
insoddisfacente che utilizza 1 bit per il segno, 4 bit per l'esponente e 4 bit per la mantissa?
\begin{itemize}
\item bit di segno: \begin{shortanswer}\item 1\end{shortanswer}
\item esponente: \begin{shortanswer}\item 1000\end{shortanswer} 
\item mantissa: \begin{shortanswer}\item 1010\end{shortanswer}
\end{itemize}
\end{cloze}





\begin{shortanswer}[points=1]{Codifica dei numeri in virgola mobile}
Quale numero \`{e} rappresentato dalla seguente sequenza di bit ottenuta dall'inesistente codifica IEEE 754 in precisione
insoddisfacente che utilizza 1 bit per il segno, 4 bit per l'esponente e 5 bit per la mantissa?
 $$ 1101110011 $$
Nota: esprimere il risultato usando il punto . come separatore tra la parte intera e quella frazionaria. Non inserire alcun carattere di spaziatura.
\item[] -25.5
\item[] -25,5
\end{shortanswer}

\begin{shortanswer}[points=1]{Codifica dei numeri in virgola mobile}
Quale numero \`{e} rappresentato dalla seguente sequenza di bit ottenuta dall'inesistente codifica IEEE 754 in precisione
insoddisfacente che utilizza 1 bit per il segno, 4 bit per l'esponente e 5 bit per la mantissa?
 $$ 0110000010 $$
Nota: esprimere il risultato usando il punto . come separatore tra la parte intera e quella frazionaria. Non inserire alcun carattere di spaziatura.
\item[] +34
\item[] +34.0
\item[] +34,0
\item[] 34
\item[] 34.0
\item[] 34,0
\end{shortanswer}



\begin{shortanswer}[points=1]{Codifica dei numeri in virgola mobile}
Quale numero \`{e} rappresentato dalla seguente sequenza di bit ottenuta dall'inesistente codifica IEEE 754 in precisione
insoddisfacente che utilizza 1 bit per il segno, 3 bit per l'esponente e 3 bit per la mantissa?
$$ 1101011 $$
Nota: esprimere il risultato usando il punto . come separatore tra la parte intera e quella frazionaria. Non inserire alcun carattere di spaziatura.
\item[] -5.5
\item[] -5,5
\end{shortanswer}

\begin{shortanswer}[points=1]{Codifica dei numeri in virgola mobile}
Quale numero \`{e} rappresentato dalla seguente sequenza di bit ottenuta dall'inesistente codifica IEEE 754 in precisione
insoddisfacente che utilizza 1 bit per il segno, 3 bit per l'esponente e 3 bit per la mantissa?
$$ 0110010 $$
Nota: esprimere il risultato usando il punto . come separatore tra la parte intera e quella frazionaria. Non inserire alcun carattere di spaziatura.
\item[] 10
\item[] 10.0
\item[] 10,0
\item[] +10
\item[] +10.0
\item[] +10,0
\end{shortanswer}


\begin{shortanswer}[points=1]{Codifica dei numeri in virgola mobile}
Quale numero \`{e} rappresentato dalla seguente sequenza di bit ottenuta dall'inesistente codifica IEEE 754 in precisione
insoddisfacente che utilizza 1 bit per il segno, 4 bit per l'esponente e 4 bit per la mantissa?
$$ 110100001 $$
Nota: esprimere il risultato usando il punto . come separatore tra la parte intera e quella frazionaria. Non inserire alcun carattere di spaziatura.
\item[] -8.5
\end{shortanswer}

\begin{shortanswer}[points=1]{Codifica dei numeri in virgola mobile}
Quale numero \`{e} rappresentato dalla seguente sequenza di bit ottenuta dall'inesistente codifica IEEE 754 in precisione
insoddisfacente che utilizza 1 bit per il segno, 4 bit per l'esponente e 4 bit per la mantissa?
$$ 010110101 $$
Nota: esprimere il risultato usando il punto . come separatore tra la parte intera e quella frazionaria. Non inserire alcun carattere di spaziatura.
\item[] 21
\item[] 21.0
\item[] 21,0
\item[] +21
\item[] +21.0
\item[] +21,0
\end{shortanswer}


\end{quiz}
\end{document}
